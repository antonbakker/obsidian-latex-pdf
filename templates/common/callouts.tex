% File: templates/common/callouts.tex
% Purpose: Provide LaTeX environments for Obsidian-style callouts (info,
% warning, tip, etc.) so that the plugin can map markdown callouts to
% consistent LaTeX blocks. These definitions are intentionally visual and
% self-contained so templates like "article" do not need to be modified to
% change callout appearance.

\usepackage[most]{tcolorbox}

% Base tcolorbox settings shared by all callouts.
% We increase before/after skip to create a bit more whitespace around each
% callout so that the horizontal rules and icon do not visually collide with
% surrounding paragraphs.
\tcbset{
  enhanced,
  sharp corners,
  boxrule=0pt,
  frame hidden,
  colback=white,
  left=2mm,
  right=2mm,
  top=1.0ex,
  bottom=1.0ex,
  before skip=1.75em,
  after skip=1.75em,
}

% Helper: severity-specific colours
\newcommand{\calloutInfoColor}{blue!60!black}
\newcommand{\calloutWarningColor}{yellow!60!black}
\newcommand{\calloutTipColor}{green!60!black}
\newcommand{\calloutNoteColor}{gray!60!black}

% Helper: simple text icons (kept package-light; colours vary by severity)
% Icons and all header text are rendered without bold; colour encodes severity.
\newcommand{\calloutIconInfo}{\textcolor{\calloutInfoColor}{I}}
\newcommand{\calloutIconWarning}{\textcolor{\calloutWarningColor}{!}}
\newcommand{\calloutIconTip}{\textcolor{\calloutTipColor}{\checkmark}}
\newcommand{\calloutIconNote}{\textcolor{\calloutNoteColor}{N}}

% Generic implementation macro: draws thin horizontal rules above and below
% the content and supports an optional description.
% Mapping from markdown to layout:
% - Situation 1: `> [!type] text`\
%   -> icon + type label on the first line, then text body, coloured by severity.
% - Situation 2: `> [!type]` / next lines = text\
%   -> same as situation 1 (icon + type label, then text body).
% - Situation 3: `> [!type] description` / next lines = text\
%   -> icon + free-form description on the first line, then text body.
%
% In all cases:
% - The colour of the icon, header/label, and horizontal lines is determined by
%   the callout severity.
% - No bold text is used; emphasis is purely via colour and layout.
\newenvironment{callout-generic}[4][]{%
  % #1: optional user description (may be empty).
  % #2: colour macro (e.g. \calloutInfoColor).
  % #3: icon macro (e.g. \calloutIconInfo).
  % #4: default label text for the severity (e.g. "Info").
  \begin{tcolorbox}[
    borderline north={0.4pt}{0pt}{#2},
    borderline south={0.4pt}{0pt}{#2},
    breakable,
  ]%
  % Decide whether a description was explicitly provided.
  \def\callout@title{#1}%
  \ifx\callout@title\@empty
    % No explicit description: show icon + default severity label.
    {#3}\hspace{0.75em}{\textcolor{#2}{#4}}\par\vspace{0.75ex}%
  \else
    % Description present: show icon + description (no bold), coloured by severity.
    {#3}\hspace{0.75em}{\textcolor{#2}{#1}}\par\vspace{0.75ex}%
  \fi
}{%
  \end{tcolorbox}%
}

% Public environments used by the plugin. They are thin wrappers around the
% generic implementation with severity-specific colours, icons, and labels.

% Info / note callout
\newenvironment{callout-info}[1][]{%
  \begin{callout-generic}[#1]{\calloutInfoColor}{\calloutIconInfo}{Info}%
}{%
  \end{callout-generic}%
}

% Warning callout
\newenvironment{callout-warning}[1][]{%
  \begin{callout-generic}[#1]{\calloutWarningColor}{\calloutIconWarning}{Warning}%
}{%
  \end{callout-generic}%
}

% Tip / success callout
\newenvironment{callout-tip}[1][]{%
  \begin{callout-generic}[#1]{\calloutTipColor}{\calloutIconTip}{Tip}%
}{%
  \end{callout-generic}%
}

% Generic fallback callout
\newenvironment{callout-note}[1][]{%
  \begin{callout-generic}[#1]{\calloutNoteColor}{\calloutIconNote}{Note}%
}{%
  \end{callout-generic}%
}
